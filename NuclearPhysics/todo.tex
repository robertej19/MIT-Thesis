https://en.wikipedia.org/wiki/Giant_resonance

Review basic vertix

https://www.youtube.com/watch?v=Sk2FzkzhnaQ&list=RDCMUCIVaddFslWk1TFoKNrvh99Q&index=19 DR PHYSICS A

Read “Probing the core of the strong nuclear reaction”

Ji & Jaffe Sum rules

What is nuclear shadowing / gluon shadowing

Need concise description of EMC

In the Plane Wave Impulse Approximation (PWIA),
an electron transfers a single virtual photon with momentum
q and energy (sometimes written !) to a single
proton, then leaves the nucleus without reinteracting
and can thus be described by a plane wave (see Fig. 1).

Why (/when) is PWIA important?

Nilsson model / nilsson modes
Kelson garvey mass formula
Neutron skin effect

How is nucleon / nuclei potential measured?
How to probe hard repulsive core of NN interaction

What is a spectral function? Spectral Decomposition?

What were the 3 explanations for EMC effect?
What is the giant resonance?

How is radius of lead, uranium, gold atoms measured
How many form factors needed per spin, and why does a neutrino have 3 FF?

How is spin parity measured
Evidence of tensor force interaction among nucleons – Deuteron! (Check this!)

Fermi momentum of carbon? Fermi momentum of lead? Why is Fermi momentum important?

Review m-scheme
Bohr-mottelson models
Peltier cooled HP-Ge
Anti-compton triggering
1 gamma enters a scint, how many electrons out of PMT?Where is the proportionality?
Why is plastic used for BAND?

Angular form is preserved across energy regimes for ep cross section!! - this note was from richard, e.g. you replace form factors with structure functions

E,e’p scattering shows the shell model really works. Evidence that other models work?

Examine more closely the electron - proton scattering diagram. From the electron photon vertex, you immediately get from energy conservation that Q2 = 4EE'sin2(theta/2) (work this out!) Also somehting about this is only valid due to angular momentum preservation for 1 photon exchange. 
What do you get at the photon proton vertex? W2 = (q + p)2 
We get quark PDFs from the F2 structure function. How do we get gluon PDFs? Can we probe the gluon directly? Does it coulple to the photon? No! We assume QCD evolution equations (e.g. DGLAP) and gluon dist comes out. 
Criteria for DIS? Q2 greter than 1, W2 greater than 4. 
What causes the PDF of 3 valence quarks not to just be delta function at 1/3 of the proton momentum? The fact that quarks are dynamic, uncertainty principle, momentum smearted out!
Half the momentum of protons come from gluons, about
Proton spin defined by G1 structure function, needto read reviewby Christine adiala. 30 percetn of proton spin from quarks, 30 percent from gluons, 40 percent from OAM. 
EMC is a statement about binding and momentum (nuclear properties) affecting partonic properties. 
Read the overview of EIC

Olympus / TPEX
Ask Loyd to explain y-scaling cuz I don’t entirely eunderstand it

Dalitz plot 

Krane Schmidt lines 126.

Roper resonance

What is partial wave analysis

Total isospin can be defined as T = 1/2(Z-N). Nuclei with N and Z exchanged are called mirror nuclei and should have a similar set of states. Isospin symmetry is broken as their masses differ by 0.14\%, interactions are not equal, and protons have charge, so Coulomb interaction is a major source of isospin symmetry breaking. Consider the relative scale of the Coulomb energy - Br73 has mass of 68 GeV, but Coulomb force is only about 20 MeV. 

. A mirror symmetry emerges from this isobaric-spin formalism: nuclei with exchanged numbers of neutrons and protons, known as mirror nuclei, should have an identical set of states7, including their ground state, labelled by their total angular momentum J and parity π . Here we report evidence of mirror-symmetry violation in bound nuclear ground states within the mirror partners strontium-73 and bromine-73. We find that a J π=5/2− spin assignment is needed to explain the proton-emission pattern observed from the T=3/2 isobaric-analogue state in rubidium-73, which is identical to the ground state of strontium-73. Therefore the ground state of strontium-73 must differ from its J π=1/2− mirror bromine-73. This observation offers insights into charge-symmetry-breaking forces acting in atomic nuclei.
         
    
From Wong pg. 251 – Deformed single particle states, produced as the eigenvecotrs of a particular Hamiltonian, are referred to as the Nilsson states or Nilsson orbitals    
    
    
Notes from Richard:
    What is strong emperial evidence of the shell model, and what is it exactly that the shell model predicts? --- from quasielastic scattering off a nucleus (basically scattering off a proton in a nucleus) and observing the electron and proton, we get teh energy of the proton, which matches up very well with what the shell model predicts.
    Phase shift - each L has a phase shift contributing to scattering, changes the sign at some point,...
    Deformations in nuclei - build off a single particle state, have rotational and vibrational modes, deformations and quadrupole moments are intimentally related, as well as being caused by a tensor force. Yukawa theory directly predicts tensor force, important for SRC, S12 comes from 1 pion exchange, 
    

    Why cant the standard model explain Dark matter? Why not neutrinos? Why not neutrons? Why not anti-matter?
    WIMPS origin?
    Dark matter needed for correct formation of galaxies in simulations: \href{http://cosmicweb.uchicago.edu/filaments.html}{link}
    
    Need to review fission decay mechanism
    
    
    
    \section{Spin 1/2 particles cannot have a quadruopole magnetic moment. Why not?}
        Spin 1/2 objects are represented by a 2x2 matrix - they can have spin states +1/2 or -1/2 along any given axis. Spin-1 objects are represented by a 3x3 matrix - their eigenstates along a given axis is +1, 0, or -1. The quadrupole moment operator can be written in terms of spin operators, e.g. $Q^{x^2-y^2} = (S^x)^2 -(S^y)^2$, which is manifestly 0 for spin-1/2 particles, but non zero for spin-1 particles. Further explanation \href{https://physics.stackexchange.com/questions/176275/why-spin-1-2-objects-doesnt-have-quadrupolar-magnetic-moment}{here}. 
    
    
    
            How to tell a 0+ state from a 2+ state? If 2+, it can emit gamma rays to decay, but a 0+ state can only decay by e+e- emission, since photons carry away at least 1 unit of angular momentum. 
            
    
    Beryllium is used because it  minimizes scattering.  Beryllium is the lightest material that has strength to hold off vacuum.
    
    
    
    
Cosmic ray flux and sources

su6 broken because delta mass is different than proton mass - SU 6 is color flavor symmetry
    
 
    
    