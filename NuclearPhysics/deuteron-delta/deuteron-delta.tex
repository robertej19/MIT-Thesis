\chapter{The Deuteron and Delta}
    \section{The Deuteron}
        \subsection{Basic properties}
            The deuteron is the \textbf{simplest example of nuclear interaction affecting partonic properties}. Why? Because the free neutron is unstable - \myquantities{a free neutron decays in $\sim$ 10 minutes}
            But, bound in a deuteron, it becomes stable. Deuterium natural abundance 0.02\%. Discovered in 1931, before the neutron was discovered. Nearly all deuterium today was produced in the Big Bang, 26 deuterium atoms per million hydrogen atoms. The NMR frequency of deuterium is 61 MHz compared to 400 MHz. \\
            \newline
            Deuterium is obtained by separation from normal water by a chemical process with hydrogen sulfide, see \href{https://en.wikipedia.org/wiki/Girdler_sulfide_process}{here} for more. D2 ice sinks in normal water. Slightly toxic to eukaryotic organisms, but not prokaryotic ones.
            \\
            Is one of only five stable odd-odd nuclides (Lithium-6, Boron-10, Nitrogen-14, and Ta-180m). Most odd-odd are unstable because they can beta decay to become even-even. Deuterium benefits from having the proton-neutron be coupled in a spin-1 state, giving a stronger nuclear attraction, which does not exist in an nn or pp stystem. If it were to, due to Pauli exclusion principle, they would have to have some other different quantum number, such as orbital angular momentum. But if either had a non-zero OAM, it would give a lower binding energy, for example. \\
            The deuteron wavefunction is a superposition of s=1, l=0 state and s=1, l=2 state. \\
            \newline
            Magnetic dipole moment is 0.857$\mu_N$, instead of the $0.88 \mu_N$ obtained by adding the moments of proton and neutron. This gives evidence of the mixing between the l=0 and l=2 state. 
            \newline
            Electric dipole is zero.\\
            \newline            
            The quadrupole is 0.29 $e fm^2$\\
            \newline
            It is in a spin triplet but an isospin singlet state. \\
            \newline
            Binding energy is 2.2 MeV.
            
        \subsection{Deuteron Wavefunction}
            spin-parity 1+, so with two spin-1/2 nucleons, only s = 0, 1 allowed. Since l restricted to J-1<L<J+1, L = 0,1,2. From parity, $(-1)^L$ must be +1 $\xrightarrow{}$ L=1 is ruled out. So only L=0, S=0 and L=2, S=1 are allowed. Deuteron is in a triplet state and is a boson. There is no singlet state. 
            
            Why is deuterium spin 1 and not spin 0?\\
        
            magnetic moment of proton, neutron, deuteron
            
            \myquantities{Deuteron binding energy - 2.2 MeV}
            \myquantities{Deuteron spin parity = $1^+$}
            \myquantities{Deuteron dipole moment}
            \myquantities{Deuteron quadrupole moment -}
            \myquantities{Proton dipole and quadrupole moment (quad should be 0 for all spin half particles}
            
            Deuteron is useful for nuclear reactors because it slows neutrons without the high neutron absorption of usual water. 
            
            For NMR spectroscopy, it is useful as its spin properites differ from normal hydrogen. 
            
            
            Deuteron spin parity 1+
Isospin T = 0
Magnetic dipole moment = 0.86 muN

Describe WF as proton, neutron, and relative orbital motion
Partiy determined by realtive orbital motino as p and n have same parity

for state wtih L, 

NN interaction has to be attractive for T =0, can be attractie or repulsive for T = 1 steate. 

"How do we know how much D-wave we have in the Deuteron?" - measure the dipole moment!

Intrinstic dipole moment of the proton and neutron, comes from the fact that they are made of quarks. There is also orbital motion of the proton.

            
    \section{The Delta States}
            What are the 4 delta states?
            Delta electric quadrupole memonet? Yes - -0.043 efm2 and magnetic octupole moment of -0.0035 efm3 for Delta+
            
            Importantly, the delta states have a mass of about 1232 MeV. The energy difference from this to a proton, about 300 MeV, is important as it represents the energy needed to flip a quark, this is called the \textbf{color hyperfine interaction}. Moreover, we can compare the energy levels of positronium and charmonium, (or any other quarkonium) and we see that there is a very similar energy level structure, but about a factor of a million difference in the magnitude of the energy level splittings. 

