\section{Time Line}
Friday 1/22: Make slides for what corrections have been made and what has not been \\
Wednesday 2/3: Going to have to make DVEP-DVMP Meeting soon - focus on what you need from the collaboration\\


Focus on getting results and production out, can worry about finesse later\\

Richard is REALLY looking for results ASAP
richard wants MC , etc
wants an MIT Monte carlo of something propaganda
MC for both

expense renewal in march 2021 - to get more money – Richard is going to push hard on computing
Need a write up by early March for Richard + computing propaganda piece by early / mid Feb.

Initivate to speed up computing:\\
ask sangbaek for skim file\\
send igor stuff to get ROOT file\\

Consider digging into code and figuring out what parts are taking the longest, then figure out how to speed them up
(notably, consider running through all hipo files and record electron, proton, and photon indicies for each event 
according to certain criteria, and then just crank through exclusivity cuts. Should be straightforward, maybe 
100 kHz processing time --> running for 2 days becomes running for 30 minutes

\section{Main Topics}
\begin{itemize}
    \item Work on extracting BSA
    \item Resolutions – smearing
    \item Helicity studies
    \item Examine the distribution of photons that are not the pion that pass exclusivity cuts
\end{itemize}


\section{Details}
\begin{itemize}
    \item Nail down exactly why it is that Andrey and my exclusivity events differ
    \item Make plot of missing mass of egg system (should be proton mass)
    \item Grab more outbending datafiles
    \item Calculate t as $t = (q-p_pi)^2$
    \item Need to draw binning on kinematics
    \item Draw min and max cut curves on xb vs q2 plot - e.g. for a maximum W2, there is a linear relationship between x and q2, and for a minimum w2, there is a x/1-x relationship between x and q2.
    \item Include info on what coatjava version is used
    \item Need to get error bars on plots
    \item Make a plot of photon angle vs electron angle (phi?)
    \item Use IceCream and Logging on python to debug
    \item Plot measured electron momentum from 5 to 10 degrees, Monte Carlo electron, etc.
    \item Run multi-streamed analysis of how using different PID methods (PID, custom, java) affects run time and how many exclusive events remain at the end
    \item Determine proper handling of multiple exclusivity candidates in one event
    \item Normalize histograms for comparing simulations to data better
    \item Look into pi0 and proton making a delta mass – search for the right cutoff in W mass
    \item Remake Plot of overlaid kinematic ranges
    \item Change missing pt plots to be independent, and also not absolute valued
    \item We can have it exit after it passing exclusivity cuts, but this is not ideal – e.g. if it finds a CD event first, but neglects an FD event
    \item Apply FX function to smear MC to improve
    \item Ask Andrey about CERN tag not working in simulations
    \item Organize writing out to a searching mysql DB and post online
    \item Incorporate in output logs what was used for PID cuts (Java, Custom, or EB only)
    \item Make message automatically write to log-analysis-runs and implement args in run\_analysis.py
    \item make plot of t vs phi
    \item make a plot of pion mass when photon and electron are in same sector
    \item make a plot of pion mass when photon and electron are not in same sector
    \item make plots comparing just EB cuts to newer cuts
    \item implement new PID cuts
    \item Compare with and without proton momentum - tag it, but dont use it to measure t and phi. Different systematics, so compare proton sytematics cd to pi systematics fd
    \item Use pi0 mass when using epgg cut, since we are already thinking we have the pi0 mass
    \item test across all 6 sectors
    \item Make previous plots all broken up on fd and cd, and without using proton momentum
    \item In outprint messages, add how many files are left to process
    \item Create some kind of SQLite / MySQL DB to handle all parameters of exclusive events, which can then be used to plot any proper combinations of variables
    \item Calculate the number of times that each part of the code is looped through
    \item Push Sangbaek about correction studies - include Andrey, FX, others, maybe pulse individually
    \item Ask Andrey about radiative effects – is that what we are seeing as the difference in the missing energy spectrum?
    \item Include why the T dependence is important (somewhere in notes)What isthe physics background for DVMP
\end{itemize}



\section{Open Questions}
\begin{itemize}
    \item How does analysis change when using custom cuts instead of normal event builder cuts?
    \item Understand what the simulation is, what is being assumed about the detector response?
    \item Why isn’t proton more pronounced in the FD phi?
    \item Why is my DVMP analysis not taking into account events where one gamma from the pion is not reconstructed?
    \item prton phi seems to be empty in CD at phi equals to 90 degrees?
    \item 2 proton theta distribution is better in CD than in permutations
    \item Is the transverse momementum exclusivity cut physically meaningful?
    \item Make beamer docs with links to other sections achronologically
    \item Resolve issue of .theta() method returning degrees but .theta(args) returning radians (look in email for emails to Sangbaek about this, middle of september 2020)
    \item What is the shadowing caused by in inbending proton theta vs proton phi plots?
    \item What percent of events make it into skim8 (ep)? 
    \item I think there might be some overcounting in the number of photons displayed in photon plots because we are permuting over photons, e.g. [1,2,4] photons - perms - each photon gets counted twice
    \item Regarding centralized information, like list of good run, efficiencies or uncertainties of common detectors, where will all this information be stored? Just in text files online somewhere?
    \item Is it possible to show the same type graphs, but as a function of the t variable. Also, even more interesting would be, a two dimensional graphs  showing the distributions of events vs Q2 and t  for a given W, e.g. 3 GeV, for both the forward and central detectors
    \item Do we worry about photons going into Central Detector?
\end{itemize}


\section{Other / NonCoding}
\begin{itemize}
    \item Put in Brandon Clary slides formatting on to base slides
    \item Respond to Valery with specific question on analysis
\end{itemize}


\section{Completed}
\begin{itemize}
     \item Post: Just to be clear, the magnetic field options for rga\_fall2018 are:
        Tor+1.0 sol-1.00
        Tor+ 1.01 sol-1.00
        Tor -1.00 sol -1.00\\
        Which corresponds to inbending and which to outbending?
    \item remove cut for BH angle and implement cut requiring photons and electrons not in same sectors
    \item make plot of photon sectors
    \item make a plot of photon sector vs electron sector
    \item make plot of electron sectors
    \item print out a list of which events numbers in 5032 passed exclusivity cuts
    \item Get new PID cuts working
    \item place cut on photon being in same sector as electron (make plot of pion mass when electron is in same sector, and when it is not)
\end{itemize}


\section{General Notes}
\begin{itemize}
    \item Systematics will dominate since not statistics limited
    \item When showing any cuts / graphs, show the graph BEFORE any cuts, and then show afterarwsd (and make labels bigger)
    \item Always keep all titles to less than 5 words: Proton 3D Substructure: New Constraints
    \item Instead of saying "pretty good" give a quantitative measure (within 1 simga, etc
    \item share with other groups 1 at a time - how to get advice from collaboration most effectively, latifa, FX, volker, valery, 1 on 1, have a dedicated BJ call
    \item make sure that when you have no signal, you do not see a signal
\end{itemize}

\section{Quote}
\begin{itemize}
    \item "When a grad student starts being able to disagree with you competently, that is the moment you graduate them" (Dont overbake or underbake them) - Janet Conrad
\end{itemize}


Note: This document is available online \href{https://latexonline.cc/compile?git=https\%3A\%2F\%2Fgithub.com\%2Frobertej19\%2FClas12AnaNote&target=main.tex&command=pdflatex\&trackId=1593973491329}{here}. It is a work in progress, and as of July 2020 is in its very early stages. It is continuously synchronized with its source overleaf document. For access to the source overleaf page, contact robertej@mit.edu. Code for this project can be found at \href{https://github.com/robertej19/analysis_code/projects/1



U channel needs central detector



