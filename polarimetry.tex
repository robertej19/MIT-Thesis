Good for beam energies between 100 MeV and 50 GeV. Polarized beam electrons are scat-
tered from other polarized electrons in a target, usually magnetized foils. Only a small
fraction of all the target electrons are polarized, so this method has a small analyzing
power. Analyzing power is exactly calculable in QED. At high beam energies, analyzing
power and scattering probability both become independently of beam energy. Maximum
analyzing power is about 80%, maximum is at 90 degrees scattering angle in C.o.M. Trans-
versely polarized target can be used to measure transverse beam polarization, but analyzing
power is only about 10%. 90 degrees C.o.M. translates to a small lab angle with each elec-
tron at half beam energy, so magnets are used to bend these electrons out to detectors.
These detectors can be, for example lead glass total absorption cherenkov counters.Since
the two electrons are corellated, can use things like time coincidence to reduce background,
although for low duty factor accelerators only one electron is required as statistics would
otherwise be too low.A main background to this process is Mott scattering with the electron
radiating off energy after scattering, appearing as a Moller electron

The scattering target is either iron or vanadium permendur (iron-cobalt alloy). Only 2 of
26 electrons in iron have their spins oriented, leading to a total analyzing power of only 6&
and transverse analyzing power of only 1%. Uncertainties in how magnetized the targets
actually are corresponds to an uncertainty in analyzing power. There are ’easy’ and ’hard’
magnetization schemes - easy does a soft magnetization, while hard uses a several tesla mag-
net to saturate the target. In principle, uncertainties on magnetization in the hard scheme
can be removed by using the Kerr magneto-optic effect, but this has not ever been imple-
mented. An important correction is due to the Levchuk effect, where due to momentum
differences between electrons in different shells, electrons scattered off of polarized electrons
are more likely to be detected than off of unpolarized electrons. Specifically, inner electrons
are unpolarized and have a large average momenta, so when struck they can fall outside the
113 TOC
acceptance of the Moller detectors, while the outer electrons, which are polarized, have a
small average momentum, and behave as expected. This is up to a 15% effect on polarization
measurements, and is currently a work in progress.

IDK what exactly JLAb does

Raster
Beam Dump