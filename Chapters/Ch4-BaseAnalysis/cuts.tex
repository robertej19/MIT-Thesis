\section{Cuts}

To arrive at a DVEP candidate event, we do the following


Code flow:

Consider a directory with n hipo files. For each hipo file, do the following.

Read each file event by event, and do the following

Check that the event has the proper databanks, and if not, go to teh next event.

Get a list of all the electrons*, protons*, and photons* in the event

*= links to most up to date PID methods

for every electron in the event (always only one, at least in the skims, but not held to be one) do the following
For every proton in the event, do the following

Calculate some basic quantities and fill histograms

for every permutation of pairs of photons in the event, do the following

calculate various kinematic quantities, and pass to see if creates a viable pion* and a viable DVEP event*

if so, fill relevant histograms and count as a DVEP event, otherwise skip to next event

viable pion: 
pion mass betwen 100 and 180 MeV
pion momentum greater than 1.5 GeV
angle (theta) between each photon and the electron to be greater than 8 degrees

viable DVEP event:
Q2 greater than 1
W greater than 2
difference between theta of missing 4-momentum and reconstructed pion less than 2 degrees
difference between missing X px and py 300 MeV each or less
Difference in missing mass squared between pion and X less than 1 GeV ** make sure this is right
difference in missing energy and X less than 1 GeV **make sure this is right

**photon cuts:
pid 22, status > 2000 (in FD or CD, not ftagger)
momentum greater than 400 MeV each

**proton cuts: pid 2212

**electron cuts: pid==1 and status < 0(negative particle