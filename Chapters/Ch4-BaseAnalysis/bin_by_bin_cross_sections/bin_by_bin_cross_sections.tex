The full differential cross section for this process can be expressed by \eqref{eq:DVPiPCrossSection_theory}

        \begin{equation*}
          \recallLabel{eq:DVPiPCrossSection_theory}
        \end{equation*}

In practice, it is easier to work with the reduced form of the cross section with the virtual photon flux factor $\Gamma$ factored out, as given by \eqref{eq:xsec_reduced}.

 \begin{equation}\label{eq:xsec_reduced}
    \frac{d^2\sigma_{\gamma^*p \rightarrow p'\pi^0}(Q^2,x_B,t,\phi_{\pi},E)}{dtd\phi} = \frac{1}{\Gamma_V(Q^2,x_B,E)} \frac{d^4\sigma_{\gamma^*p \rightarrow p'\pi^0}}{dQ^2dx_Bdtd\phi_{\pi}}.
\end{equation}

At this point, cross section results can be calculated on a bin-by-bin basis. Example results are shown for sample bins in \figref{fig:redxsec_phi}. Data points are black points with error bars (black=statistical only, red = total). The black curve is a A+B$\cos(\phi)$ +C$\cos(2\phi)$ fit to the data to allow for structure extraction, the results of which are discussed in Chapter \ref{Chapter:Ch5_IBU}. The blue curve is the result from the CLAS6 experiment \parencite{Bedlinskiy2014ExclusiveCLAS} adjusted for the difference in beam energy. The blue band surrounding the blue curve is the approximate error for the CLAS6 result, and in general shows reasonable agreement where there is kinematic overlap. Bin-by-bin methods neglect some binning effects, discussed in the next chapter. Full results for all bins are displayed in Figs. \ref{fig:combined_t0.2} - \ref{fig:combined_t1.5}.



\iffalse
 \begin{equation}\label{xsec_red_full}
    \frac{d^2\sigma_{\gamma^*p \rightarrow p'\pi^0}}{dtd\phi} = \frac{1}{\Gamma_V(Q^2,x_B,E)} \frac{N(Q^2,x_B,t,\phi_{\pi})}{\Lumi_{int}\Delta Q^2\Delta x_B\Delta t\Delta \phi} \frac{1}{\epsilon_{ACC} \delta_{RC} \delta_{Norm} Br(\pi^0 \rightarrow \gamma \gamma)}
\end{equation}

\fi

\begin{figure}[H]
    \centering

    \subfloat[$\langle x_B \rangle$=0.23,$\langle Q^2 \rangle$=1.73 $GeV^2$,$\langle t \rangle$=0.26 $GeV^2$]{\includegraphics[width=0.5\textwidth]{Chapters/Ch4-BaseAnalysis/bin_by_bin_cross_sections/singles/xqt=(0.2, 1.5, 0.2).png}}
    \hfill
    \subfloat[$\langle x_B \rangle$=0.62,$\langle Q^2 \rangle$=7.88 $GeV^2$,$\langle t \rangle$=1.24 $GeV^2$]{\includegraphics[width=0.5\textwidth]{Chapters/Ch4-BaseAnalysis/bin_by_bin_cross_sections/singles/xqt=(0.58, 7.0, 1.0).png}}

    \caption[Reduced Cross Sections Across $\phi$]{Reduced cross sections across $\phi$ at (a) low kinematics and (b) high kinematics. A sample of the CLAS6 result \parencite{Bedlinskiy2014ExclusiveCLAS} is shown in blue with an error band (in modified form to account for beam energy differences). }\label{fig:redxsec_phi}
    %\caption[Reduced Cross Sections Across $\phi$]{Reduced Cross Sections Across $\phi$}
\end{figure}




\iffalse

\begin{figure}[ht]
\centering
\includegraphics[trim={14.6cm 4cm 27.2cm 4cm},clip,width=\textwidth]{Chapters/Ch4-BaseAnalysis/bin_by_bin_cross_sections/pics_screenshots/t_2.png}
\caption[Reduced Cross Section for 0.2 $GeV^2 < t <$ 0.3 $ GeV^2$]{Reduced Cross Section for 0.2 $ GeV^2 < t <$ 0.3 $GeV^2$ in bins of $x_B$ (increasing left to right) and $Q^2$ (increasing vertically upwards). }
\label{fig:combined_t0.2}
\end{figure}


\begin{figure}[ht]
\centering
\includegraphics[trim={14.1cm 4cm 27.2cm 4cm},clip,width=\textwidth]{Chapters/Ch4-BaseAnalysis/bin_by_bin_cross_sections/pics_screenshots/t_3.png}
\caption[Reduced Cross Section for 0.3 $GeV^2 < t <$ 0.4 $ GeV^2$]{Reduced Cross Section for 0.3 $ GeV^2 < t <$ 0.4 $GeV^2$.}
\label{fig:combined_t0.3}
\end{figure}


\begin{figure}[ht]
\centering
\includegraphics[trim={14.6cm 4cm 23cm 4cm},clip,width=\textwidth]{Chapters/Ch4-BaseAnalysis/bin_by_bin_cross_sections/pics_screenshots/t_4.png}
\caption[Reduced Cross Section for 0.4 $GeV^2 < t <$ 0.6 $GeV^2$]{Reduced Cross Section for 0.4 $GeV^2 < t <$ 0.6 $GeV^2$.}
\label{fig:combined_t0.4}
\end{figure}


\begin{figure}[ht]
    \centering
    \includegraphics[trim={14.8cm 4cm 18.5cm 4cm},clip,width=\textwidth]{Chapters/Ch4-BaseAnalysis/bin_by_bin_cross_sections/pics_screenshots/t_6.png}
    \caption[Reduced Cross Section for 0.6 $GeV^2 < t <$ 1.0 $GeV^2$]{Reduced Cross Section for 0.6 $GeV^2 < t <$ 1.0 $GeV^2$.}
    \label{fig:combined_t0.6}
\end{figure}


\begin{figure}[ht]
    \centering
    \includegraphics[trim={14.8cm 4cm 18.5cm 4cm},clip,width=\textwidth]{Chapters/Ch4-BaseAnalysis/bin_by_bin_cross_sections/pics_screenshots/t_10.png}
    \caption[Reduced Cross Section for 1.0 $GeV^2 < t <$ 1.5 $GeV^2$]{Reduced Cross Section for 1.0 $GeV^2 < t <$ 1.5 $GeV^2$.}
    \label{fig:combined_t1.0}
\end{figure}

\begin{figure}[ht]
    \centering
    \includegraphics[trim={18.8cm 4cm 18.5cm 4cm},clip,width=\textwidth]{Chapters/Ch4-BaseAnalysis/bin_by_bin_cross_sections/pics_screenshots/t15.png}
    \caption[Reduced Cross Section for 1.5 $GeV^2 < t <$ 2.0 $GeV^2$]{Reduced Cross Section for 1.5 $GeV^2 < t <$ 2.0 $GeV^2$.}
    \label{fig:combined_t1.5}
\end{figure}

\fi