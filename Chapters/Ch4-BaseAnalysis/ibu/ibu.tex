Here we talk about Iterative Bayesian unfolding to not have bin migration issues

Notes on Omnifold from Anselm Vossen:
I find this reference paper - \href{https://arxiv.org/pdf/1911.09107.pdf}{arxiv}.




Yes, that is the reference. You saw that I also posted a reference to the Hera analysis. I don’t think many others have used omnifold yet, since it is quite computationally intensive etc.
My understanding is that Ben Nachman developed this and there has been follow up work by his group (if you just put his name into inspire you’ll see). E.g. I saw presentations on how to present the data. Here is a talk by Ben at a Jet workshop in 2021: \href{https://indico.bnl.gov/event/10555/contributions/54721/attachments/37368/61570/H1Measurement_EICWorkshopSeptember2021.pdf}{this paper}
 Miguel Arratia, who is also in CLAS collaborated on the H1 results. You could ask him for practical advice,

From CHATGPT:


Iterative Bayesian unfolding is a statistical technique used in particle physics and other fields to correct for detector effects and reconstruct the true distribution of a measured quantity.

When particles are measured by a detector, the measurements can be distorted by various effects such as inefficiencies, energy loss, and detector resolution. This leads to a measured distribution that can be quite different from the true distribution.

Unfolding techniques aim to invert these detector effects to estimate the true distribution from the measured one. The unfolding problem is challenging because it is ill-posed; small fluctuations in the data can lead to large fluctuations in the unfolded result.

Bayesian unfolding is based on Bayes' theorem and uses a prior probability distribution that reflects our knowledge (or assumptions) about the true distribution before the measurement. The Bayesian approach allows for the incorporation of systematic uncertainties into the unfolding process.

Iterative Bayesian unfolding is a variant of Bayesian unfolding that iteratively applies the unfolding procedure. The output of one iteration is used as the prior for the next iteration. This iteration process is repeated until the result converges.

Please note that unfolding techniques are sensitive to the chosen model and prior, and they require careful validation and error estimation. The choice between iterative Bayesian unfolding and other unfolding techniques (like matrix inversion or regularized unfolding) often depends on the specific problem and the constraints of the analysis.