\section{General Analysis Overview}



For each kinematic bin the differential cross section can be written as:

\begin{equation}
    \sigma = \frac{N_{meas}}{L \epsilon}\frac{1}{\delta}
\end{equation}

Where $\frac{N_{meas}}{L}$ is the number of events from experiment normalized by the integrated luminosity before acceptance and radiatvie corrections. $\epsilon$ = $\frac{N^{RAD}_{rec}}{{N^{RAD}_{gen}}}$ is the acceptance correction and $\delta$ is the radiative correction.



$\delta$ can be obtained by using the following:

\begin{equation}
    \delta = \frac{N^{RAD}_{gen}}{N^{NORAD}_{gen}}
\end{equation}

$\delta$ and $\epsilon$ need to be properly calculated, but for a first pass we will ignore them so we have just


\begin{equation}
    \sigma = \frac{N_{meas}}{L}
\end{equation}

We can calculate the luminosity L through the following equation

\begin{equation}
    L = \frac{N_A l \rho Q_{FCUP}}{e}
\end{equation}

Where $N_A$ is Avogadro's constant, l is the length of the target,  $\rho$ is the density of the target (liquid hydrogen), $Q_{FCUP}$ is the charge collected on the Faraday cup, and e is the charge of the electron. The values of these quantities are (ignoring uncertainties on experimental quantities for the time being):\\

$N_A$ = 6.02214 x $10^{23}$\\
l = 5 cm\\
$\rho$ = 0.07 $g/cm^3$\\
e = 1.602 x $10^-19$ Coulombs\\
$Q_{FCUP}$ - this must be measured and obtained from analysis. Typical runs at CLAS12 have an accumulated beam charge of tens to hundreds of thousands of nanoCoulumbs. 

\section{Data Pre-Processing}
    \subsection{Energy Loss Corrections}
    \subsection{Momentum Corrections}
    \subsection{Simulation:Experiment Resolution Matching}
        \subsubsection{Kinematics Correction of Experimental Data}
        \subsubsection{Smearing Simulated Data}

\section{Particle Identification}

\section{Event Selection}
    \subsection{Rigid Event Selection}
    \subsection{Classifier Based Event Selection}

\section{Luminosity}

\section{Configuration and Kinematics}

\section{Binning}

\section{Acceptance Correction}

\section{Radiative Corrections}

\section{Binning Corrections}

\section{Overall Normalization Corrections}

\section{Error Analysis}

