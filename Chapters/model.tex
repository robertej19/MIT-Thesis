\section{GK Model}

We compare the preliminary cross section to the model developed by S.V. Goloskokov and P. Kroll \cite{Goloskokov2010}. This model uses the handbag model to produce theoretical curves for specified sets of kinematic points. This model was implemented in the PARTONS framework \cite{Berthou2018} and was also used in the published CLAS6 result to compare with their experimental cross section, reproduced below in figure \ref{fig:oldres} \cite{Bedlinskiy2014}.

\begin{figure}[hbt]
	\centering
	\includegraphics[page=6,width=0.6\linewidth]{clas6comp.jpg}
\end{figure}\label{fig:oldres}

To validate the model, we ran the implementation to generate curves and compared to the published CLAS6 result. We observed that the sigma T and sigma L terms were comparable, but not exactly the same, as the 2014 published results, while the sigma TT term was significantly different. It is believed that these differences are due to improvements in the model made in the past 8 years. Figure \ref{fig:oldres2} shows one example bin of this comparison, where the color of the curves is matched to the corresponding color of the structure functions.


The parameters for the GK model were taken from 

Their formulation is quite similar albeit with important differences. First of all, the pi^0 production does not include the so-called pion-pole contribution (see Eq. 4.39 - 4.42 in my thesis). Moreover, their handbag contributions are slightly different. Their differences at the handbag level are discussed in Eq. 4.37 and 4.38 in my thesis. 

The parameters of GPDs and their t-slope are not completely the same. At the time I wrote the code, I took the most updated parameters from P. Kroll (or parameters that he thought would best describe the JLab kinematics in Fig. 3 of arxiv.org/pdf/2007.15677.pdf). So, I would not expect the same curves just because the GPDs in those works have different parameters. 

Lambda is defined as: https://en.wikipedia.org/wiki/Källén_function

Description of GK by Kemal Thesis:

The Goloskokov-Kroll model has been phenomenologically successful (inlcude links showing this).  The description is based on QCD factorization theorems. In factorizable processes, the amplitudes can be written as a convolution of a hard scattering which is computable in pQCD, and a soft non perturbative part parameterized by GPDs. Chiral-even GPDs are accessible through DVCS where factorization was proven (CITE). Chiral-odd GPDs can be accessed at subleading twist through Deeply Virtual Meson Production if one assumes an effective handbag mechanism, as descried by the GK model. 
QCD factorization theorem for DVMP process has only been proven for longitudinally polarized photons, and also that the cross section is suppressed by a power of 1/Q for transversley polarized photons. 
The GK model computes contributions from transversely polarized photons in the handbag mechanism as a twist-3 effect in which teh soft part of the process is parameterized in terms of Chiral Odd GPDs.
Several GK model parameters implemented in teh PARTONS framework differ from teh parameters used in refrences. The GK model parameters implemented are used in two different publicatoins
The GK model, under the assumption of flavor-symmetric sea GPDs, only valence quark GPDs Htilda Etilda Ht and Etildat are needed to describe teh process in the kinematical region of large Q2 but small zai and t. GPDs in teh GK model are constructed from double distribtuions as follows, which can be integrated analytically, and the GPDs can be expressed in teh following form:


From Easy as Pi:
Among the many important consequences is the fact that differently from both inclusive and semi-inclusive processes, GPDs can in principle provide essential information
for determining the missing component to the nucleon longitudinal spin sum rule, which
is identified with orbital angular momentum. A complete description of nucleon structure
requires, however, also the transversity (chiral odd/quark helicity-flip) GPDs, HT (x, ξ, t),
ET (x, ξ, t), HeT (x, ξ, t), and EeT (x, ξ, t) [1]. J


\begin{figure}[hbt]
	\centering
	\includegraphics[page=6,width=0.6\linewidth]{2022_vs_2014_GK_model.jpg}
\end{figure}\label{fig:oldres2}

Finally, we compare the preliminary CLAS12 reduced cross section to the predictions from the GK model. Sample plots are shown below. Agreement is close but not exact. The functional form is as expected. It is unclear if the offset between the CLAS12 fit and the GK model is due to a model discrepancy, or an absolute normalization uncertainty in the CLAS12 calculation. More quantitative statments will be made when uncertainties and correction factors in the CLAS12 work are better understood.
\begin{figure}[hbt]
	\centering
	\includegraphics[page=6,width=0.45\linewidth]{reduced_xsec_1.5_2_0.2_0.25_0.2_0.3.png}
	\includegraphics[page=6,width=0.45\linewidth]{reduced_xsec_1.5_2_0.25_0.3_0.2_0.3.png}
\end{figure}\label{fig:oldres}


\begin{figure}[hbt]
	\centering
	
\end{figure}\label{fig:oldres}


