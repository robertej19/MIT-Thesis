
% NOTE:
% These templates make an effort to conform to the MIT Thesis specifications,
% however the specifications can change. We recommend that you verify the
% layout of your title page with your thesis advisor and/or the MIT 
% Libraries before printing your final copy.
\title{Measurement of the Deeply Virtual Neutral Pion Electroproduction Cross Section at the Thomas Jefferson National Accelerator Facility at 10.6 GeV}

\author{Robert Johnston}
% If you wish to list your previous degrees on the cover page, use the 
% previous degrees command:
%       \prevdegrees{A.A., Harvard University (1985)}
% You can use the \\ command to list multiple previous degrees
%       \prevdegrees{B.S., University of California (1978) \\
%                    S.M., Massachusetts Institute of Technology (1981)}
\department{Department of Physics}

% If the thesis is for two degrees simultaneously, list them both
% separated by \and like this:
% \degree{Doctor of Philosophy \and Master of Science}
\degree{Interdisciplinary PhD in Physics and Statistics}

% As of the 2007-08 academic year, valid degree months are September, 
% February, or June.  The default is June.
\degreemonth{June}
\degreeyear{2023}
\thesisdate{May 5, 2023}

%% By default, the thesis will be copyrighted to MIT.  If you need to copyright
%% the thesis to yourself, just specify the `vi' documentclass option.  If for
%% some reason you want to exactly specify the copyright notice text, you can
%% use the \copyrightnoticetext command.  
%\copyrightnoticetext{\copyright IBM, 1990.  Do not open till Xmas.}

% If there is more than one supervisor, use the \supervisor command
% once for each.
\supervisor{Richard Milner}{Professor of Physics}

% This is the department committee chairman, not the thesis committee
% chairman.  You should replace this with your Department's Committee
% Chairman.
\chairman{Lindley Winslow}{Associate Department Head of Physics}

% Make the titlepage based on the above information.  If you need
% something special and can't use the standard form, you can specify
% the exact text of the titlepage yourself.  Put it in a titlepage
% environment and leave blank lines where you want vertical space.
% The spaces will be adjusted to fill the entire page.  The dotted
% lines for the signatures are made with the \signature command.
\maketitle

% The abstractpage environment sets up everything on the page except
% the text itself.  The title and other header material are put at the
% top of the page, and the supervisors are listed at the bottom.  A
% new page is begun both before and after.  Of course, an abstract may
% be more than one page itself.  If you need more control over the
% format of the page, you can use the abstract environment, which puts
% the word "Abstract" at the beginning and single spaces its text.

%% You can either \input (*not* \include) your abstract file, or you can put
%% the text of the abstract directly between the \begin{abstractpage} and
%% \end{abstractpage} commands.

% First copy: start a new page, and save the page number.
\cleardoublepage
% Uncomment the next line if you do NOT want a page number on your
% abstract and acknowledgments pages.
% \pagestyle{empty}
\setcounter{savepage}{\thepage}
\begin{abstractpage}
% $Log: abstract.tex,v $
% Revision 1.1  93/05/14  14:56:25  starflt
% Initial revision
% 
% Revision 1.1  90/05/04  10:41:01  lwvanels
% Initial revision
% 
%
%% The text of your abstract and nothing else (other than comments) goes here.
%% It will be single-spaced and the rest of the text that is supposed to go on
%% the abstract page will be generated by the abstractpage environment.  This
%% file should be \input (not \include 'd) from cover.tex.



%{\Huge $\braket{\Psi | \Phi}$}
%{\fontsize{180}{240} \selectfont  $\braket{\Psi | \Phi}$}

Deeply virtual exclusive reactions provide unique channels to study both transverse and longitudinal properties of the nucleon simultaneously, allowing for a 3D image of nucleon substructure. This presentation will discuss work towards extracting an absolute cross section for one such exclusive process, deeply virtual neutral pion production, using 10.6 GeV electron scattering data off a proton target from the CLAS12 experiment in Jefferson Lab Hall B . This measurement is important as exclusive meson production has unique access to the chiral odd Generalized Parton Distributions, and is also a background for other exclusive processes such as Deeply Virtual Compton Scattering, making the determination of this cross section crucial for other exclusive analyses.

\end{abstractpage}

\clearpage

\iflong{\section*{Acknowledgments}

Richard Milner is an outstanding research advisor and I am deeply grateful to have carried out this effort as part of his group. 
Janet Conrad and Will Detmold for being on committee

The list of people who have been part of path towards this result is, to me, more impressive than the result itself. I am not able to thank everyone here (the list is over 150 people, and that is still likely an underestimate)


JLab/JSA Graduate Fellowship and APS DNP Travel Fund for providing support for
my Ph.D. study and dissemination and the FRANK FIRST YEAR FELLOWS




Grateful for Richard to allow me time to work on these problems

Alan Guth	
Alex Diaz	
All JLab hockey guys	
Andrew Cantwell	
Andrew Cantwell for liking every single one of my posts	
Andrey Kim	
Anna Maria Convertino	
Axel Schmidt	
Benedikt Maier	
Bolek	
Brandon BDOTFRESH	
Brandon Clary	
Brandon Kriesten	
Brett Estrella	
Calvin Mealer	That big guy that threatened to eat me
Cathy 	
Cathy Modica	
Charles Epstein	
Chiara	
Chris Tschalaer	
Chris Vidal	
Christoph Paus	
Chuckwloka doctor	
Coach Deli	
Coby Unger	
Cole Smith	
Connor Walsh	
Dan Carman	
Dan Lacross	
David Lawerence	
David Riser	
Derek Glazier	
Doug Hasell	
Ed Pancoast	
Efrain Segarra	
Eileene Milner	Spelling
Elsye Luc	
Elton Smith	
Ernie Ihloff	
Eugene Pasyuk	
Field Rose Rogers	
Florian Hauenstein	
Francois-Xavier Girod	
Frank Taylor	
Fridericke Jentoft	
Gagik Gavalian	
Garnette	
Haley	
Harut Avadkian	
Hayami	
Igor Korover	
Inky Johnson	
Ivica Friščić	
Jack McGlashing	
Jake Reed	
Jan Berneauer	
Jan Berneaur	
Jen	
Jim Kelsey	
Joe Griffin, for pulling me through stats	
Joe the UMass Lab technician	
Jon / people from BUTPC	
Kandice Carter at JLab publications	
Karen Dow	
Karen Down	
Kyle Shields	
Kyungseon Joo	
Latifa Elouadrhiri	
Laura Hild at JLab scicomp	
Lauren Saragosa	
Lee (the guy that died)	
Lindley WInslow	
Logiduice	
Loyd Waites	
Mac Mestayer	
Madavin	
Marco Battaglieri	
Marco Contalbrigo	
Matt Pietrek	
Matt Pietrek	
Matt Sheerer	
Maurizio Ungaro	
Maurizio Ungaro	
Max Goncharov	
Maxime Defurne,	
McEvoy	
Mike Williams	
Mis Kish	
Mr Fuhr	
Mr Noon	
Mr Phillips	
Mr Pineda	
Mr Thompson	
Mrs. Messina	
Mrs. Rice	
Nathan Baltzell	
Nick Buzinsky	
Nick Cambi	
Nick Koulopoulos	
Nick Markov	
Or Hen	
Patrick Moran	
Paul Acosta	
Paul Stoler	
Peter FIsher	
Phil Regollet	
Possum	
Raffaella de Vita	
Richard Milner	
Rory Miskimen	
Ross Corliss	
Sangbaek Lee	
Sangbaek Lee	
Seth Loyd	
Skylar Whitney	
Snr Mastandrea	
Stan Weisser	
Stefan Diehl	
Stepan Stepanyan	
Steve Steadman	
Sydney MIller	
Tami Paluca	
The Two SURA host people	
Thomas Frank 	First Year fellows
Tom Boettcher	
Valery Kubarovsky	
Veronique Ziegler	
Volker Burket	
Wasalu Jaco	
Xiangdong Ji	
Xiaqing Lee	
Yimin Wang	
Yunjie Yang	
Yury Polansky	
Zhangqier Wang	

\fi


 