\section{Exploring Structure through Scattering}\label{ch1:sec1:background}

Humanity interested in the structure of reality, atomic theory, subatomic, subnuclear, structure of this thesis.  The rest of this chapter discusses these advancments, and describes the theoretical background for this experiment. Chapter 2 descriptes the experiment, CHapter 5 the analysis, etc. Chapter 8 summarizes this work and discusses how this analysis will proceed in the near future and a path for future experiments. The appendix includes numerous technical details and upplemental plots.

As we increase the resolution (resolve features over smaller spatial distances), what we see is dependent on what resolution scale we are at. An exception to this case is if we are investigating point-like particles, which would have an identical response across all resolution scales

GPDs combine the kinematics of both elastic form factors and parton distributions. Andrey references 5, 6, and 7

\subsection{Discovery of the Proton}
Yimin has good references for this part
\subsection{Electron as a Nucleon Probe}

Include discussion of wavelength and momentum transfer (but in practice, limiting factor to resolution is lens system %https://advanced-microscopy.utah.edu/education/electron-micro/%

1961 hofstadter nobel prize (Andrey reference 1)
Andrey refrence 2 Friedman Kendall Tayleor scaling

Andrey thesis good write ups

Sangbaek:
proton has properties
Describe with QFT, words about QCD
role of experiment in studying proton structure
- proton discovery
- neutron discovery
- pointlike consitutents
- development of quark model
- scaling behavior and asymptotic freedom
- details about scattering experiments from first principles
- words about elastic scattering (mott scattering)
- Plot of elastic form factors, discussion of Ge and Gm, Rosenbluth formula
- Mention TPEX
- discussion of inelastic scattering
- Structure funcitons
- Discussion of spin and sum rules

Now discussion of DVEP:
GPDs, Wigner functions
relationships all the way around
some annoying math
Handbag diagram
Lepton hadron plane
Status of experiments and future (EIC mapping)




%Maybe someone will tell you to include something about the standard model



invariant mass is W 




Understanding the structure of matter has been a fundamental research pursuit for centuries. 


Proton not a point mass - it has structure



    \subsection{Structure of our World}
    - atomic theory
    - discovery of proton / nucleon
    - proton structure measurements
    - lepton scattering experiments, HERA, etc


\section{Theoretical Background}
    Discussion of Form Factors and genralized proton structure, Wigner Functions
    
    \subsection{GPDs and Deeply Virtual Scattering}

          The cross section for DV$\pi^0$P has been theoretically linked to GPDs, which describe the 3D structure of the nucleon.

    
     \begin{equation}\label{DVPiPXCrossSection}
           \frac{d^4\sigma_{\gamma^*p \rightarrow p'\pi^0}}{dQ^2dx_Bdtd\phi_{\pi}} =
         \Gamma (Q^2, x_B, E)
         \frac{1}{2\pi}
         \left\{ \left(  \textcolor{sigmaT}{\frac{d\sigma_T}{dt}}+\epsilon  \textcolor{sigmaL}{\frac{d\sigma_L}{dt}} \right)+
         \epsilon cos(2\phi)  \textcolor{sigmaTT}{\frac{d\sigma_{TT}}{dt}} + 
         \sqrt{2\epsilon(1+\epsilon)} cos(\phi)  \textcolor{sigmaLT}{\frac{d\sigma_{LT}}{dt}} \right\}
     \end{equation}
     
    There are 8 GPDs, 4 correspond to helicity conserving (chiral even) processes and 4 correspond to are helicity flipping (chiral odd) processes: \GPDH,  \GPDE,  \GPDHtilde,  and \GPDEtilde for chiral even, and \GPDHT,  \GPDET,  \GPDHTtilde, and \GPDETtilde \\
\GPDETbar = 2*\GPDHTtilde+\GPDET is commonly used

    
    \begin{table}[H]
        \centering
        \begin{tabular}{@{} *{4}{c} @{}}
                \headercell{Nucleon \\ Polarization} & \multicolumn{3}{c@{}}{Quark Polarization}\\
                \cmidrule(l){2-4}
                & U & \textcolor{white}{lllll}L & T    \\ 
                \midrule
                  U  & \GPDH &                                   &  \GPDETbar \\
                  L  &                    &  \textcolor{white}{llll}\GPDHtilde &                                   \\
                  T  & \GPDE &                                   &  \GPDHT,\GPDHTtilde \\
            \end{tabular}\\
            
    \end{table}
    


    
    Why do the scrtucture combine in the way they do with the coefficents of cos phi terms and epsilons?
    
    And the structure functions can be written as:

    \begin{equation}
         \textcolor{sigmaL}{\frac{d\sigma_{L}}{dt}} = 
        \frac{4\pi\alpha}{kQ^2}\left\{ \left( 1 - \xi^2 \right) 
        \lvert \langle \GPDHtildeEQ \rangle \rvert ^2 
        -2\xi^2 \Re \left[  \langle \GPDHtildeEQ \rangle ^* \langle \GPDEtildeEQ \rangle    \right] - \frac{t'}{4m^2}\xi^2
        \lvert \langle \GPDEtildeEQ \rangle \rvert ^2  \right\}
    \end{equation} 
  
    \begin{equation}
        \textcolor{sigmaT}{\frac{d\sigma_{T}}{dt}} = 
        \frac{2\pi\alpha \mu_{\pi}^2}{kQ^4}
        \left\{ \left( 1 - \xi^2 \right) 
        \lvert \langle \GPDHTEQ \rangle \rvert ^2
        - \frac{t'}{8m^2}
        \lvert \langle \GPDETbarEQ \rangle \rvert ^2  \right\}    
    \end{equation} 
    
    \begin{equation}
        \textcolor{sigmaLT}{\frac{d\sigma_{LT}}{dt}} = 
        \frac{4\pi\alpha \mu_{\pi}}{\sqrt{2}kQ^3}
        \xi\sqrt{1-\xi^2}
        \frac{\sqrt{-t'}}{2m}
        \Re \left\{ 
         \langle \GPDHTEQ \rangle ^*
        \langle \GPDEtildeEQ \rangle   
        \right\}
     \end{equation} 
    
    
    \begin{equation}
        \textcolor{sigmaTT}{\frac{d\sigma_{TT}}{dt}} = 
        \frac{4\pi\alpha \mu_{\pi}^2}{kQ^4}
        \frac{-t'}{16m^2}
        \langle \GPDETbarEQ \rangle^2   
    \end{equation} 
        
    

    and epsilon is... 
    
    and t' stands for.. t - $t_0$ where $t_0 = \frac{-4m^2\xi^2}{1-\xi^2}$
    
    
    Where the skewness parameter is $\xi = \frac{x_B}{2-x_B}$ or whatever
    

    The bracket $\langle \tilde{F} \rangle$ is the convolution of a GPD and an appropriate subprocess amplitude:
    $
    \langle \tilde{F} \rangle =  \Sigma_{\lambda} \int_{-1}^{1} d\bar{x}H_{0\lambda,0\lambda}\left( \bar{x}, \xi, Q^2, t=0  \right)\tilde{F}\left( \bar{x}, \xi, Q^2, t  \right)\   
    $ 


    Where $\lambda$ is the unobserved helicites of the partons participaing in the subprocess
    
    
    
    
    And k is the phase space factor given as 
     \scalebox{0.7}{%
     $           k = 16\pi \left( W^2 -m^2)\sqrt{\Lambda(W^2,-Q^2,m^2)} \right)$ 
    }
    
        Where $\Lambda(W^2,-Q^2,m^2)$ is the Källén function and $\mu_{pi}$ is the reduced pion mass given as 
       \scalebox{0.7}{%
      $         \mu_{\pi^0} = \frac{m^2_{\pi^0}}{m_u+m_d}$ 
    }
    $m_u$ and $m_d$ are respective masses of up and down quarks.


        Include proton pressure distribution plot

      

    \begin{equation}
                 \Gamma (Q^2, x_B, E) = \frac{\alpha}{8\pi} \frac{Q^2}{m^2_pE^2}\frac{1-x_B}{x_B^3}\frac{1}{1-\epsilon}
    \end{equation}
In addition to collinear momentum distribution of partons inside the
nucleon, GPDs also encode the distribution of partons in the plane transverse to
the nucleons momentum in the infinite momentum frame [58]. Moreover, their
relation to energy-momentum tensor (EMT) form factors allow us to access the
EMT densities, the distribution of energy, angular momentum, pressure, and shear
forces inside the nucleon [15].

Only valence quarks contribute electroproduction of uncharged pions.

    \subsection{The Handbag Approach}

    

 DVMP is sensitive to chiral odd GPDs, distinguishing it from DVCS as a GDP probe because why? Because something involving photon helicity and pion helicity, I forget exactly though
 
    \subsection{The Goloskokov-Kroll Model}
    
\section{Overview of Experimental Status}
    \subsection{Previous Experimental Results}


    \subsection{Overview of this Analysis}
    
Hi \cite{Bedlinskiy2014} see more in section ~\ref{ch1:sec1:background}
just a test



\iffalse
    \subsection{Unused}
        The following is from a paper on pi+ production. This formula is equvalent to the normal convetion (not using W as a variable)
        
         \begin{equation}\label{xsec}
             \frac{d^4\sigma_{\gamma^*p \rightarrow p'\pi^0}}{dQ^2W^2dtd\phi_{\pi}} =
             \frac{\alpha (W^2-m^2)}{16\pi^2 E^2_L m^2 Q^2 (1-\epsilon)}
             ((\frac{d\sigma_T}{dt}+\epsilon\frac{d\sigma_L}{dt})+
             \epsilon cos(2\phi) \frac{d\sigma_{TT}}{dt} + \sqrt{2\epsilon(1+\epsilon)}cos(\phi)\frac{d\sigma_{LT}}{dt})
        \end{equation}
        
        Comparing the two, we have a difference in the prefactor of:
        
        0.3894 * 1E6 * $\frac{1}{16\pi(W^2-m_p^2)\sqrt{W^4 + (Q^2)^2+m_p^4+2W^2Q^2-2W^2m_p^2+2Q^2m_p^2}}$
        
        This factor is accounted for by the Kallen function phase space term.
        
        
        
        
        t' stands for.. t - $t_0$ where $t_0 = \frac{-4m^2\xi^2}{1-\xi^2}$
        
           Where the skewness parameter is $\xi = \frac{x_B}{2-x_B}$ 
           
        
        
            Where $\lambda$ is the unobserved helicites of the partons participaing in the subprocess
            
         and epsilon is... 
            
            
            
            
         
               
            And k is the phase space factor given as 
             \scalebox{0.7}{%
             $           k = 16\pi \left( W^2 -m^2)\sqrt{\Lambda(W^2,-Q^2,m^2)} \right)$ 
            }
            
                Where $\Lambda(W^2,-Q^2,m^2)$ is the Källén function and $\mu_{pi}$ is the reduced pion mass given as 
               \scalebox{0.7}{%
              $         \mu_{\pi^0} = \frac{m^2_{\pi^0}}{m_u+m_d}$ 
            }
            $m_u$ and $m_d$ are respective masses of up and down quarks.
            
             Where $\Gamma (Q^2, x_B, E)$ is the virtual photon flux, is 
             \scalebox{0.7}{%
                $         \Gamma (Q^2, x_B, E) = \frac{\alpha}{8\pi} \frac{Q^2}{m^2_pE^2}\frac{1-x_B}{x_B^3}\frac{1}{1-\epsilon}$ 
            }

\fi
