\section{Background - Structure of the Proton}\label{ch1:sec1:background}

\section{Deeply Virtual Neutral Pion Production}
    \subsection{The Handbag Approach}
    \subsection{The Goloskokov-Kroll Model}
    \subsection{Status of Measurements}
    \subsection{Analysis Overview}
    
Hi \cite{Bedlinskiy2014} see more in section ~\ref{ch1:sec1:background}
just a test

\iffalse


From paper on understanding pi+ production, we have:

 \begin{equation}\label{xsec}
     \frac{d^4\sigma_{\gamma^*p \rightarrow p'\pi^0}}{dQ^2W^2dtd\phi_{\pi}} =
     \frac{\alpha (W^2-m^2)}{16\pi^2 E^2_L m^2 Q^2 (1-\epsilon)}
     ((\frac{d\sigma_T}{dt}+\epsilon\frac{d\sigma_L}{dt})+
     \epsilon cos(2\phi) \frac{d\sigma_{TT}}{dt} + \sqrt{2\epsilon(1+\epsilon)}cos(\phi)\frac{d\sigma_{LT}}{dt})
\end{equation}

Comparing the two, we have a difference in the prefactor of:



0.3894 * 1E6 * $\frac{1}{16\pi(W^2-m_p^2)\sqrt{W^4 + (Q^2)^2+m_p^4+2W^2Q^2-2W^2m_p^2+2Q^2m_p^2}}$




From Kemal thesis intro:

Even though it is accounted for, you should have an answer to the question "How many times does the process end up with only 1 photon found" since you are comparing it to DVCS several times

Fix CLAS12 slide to show better how particles are found like how Sangbaek did it and include specs (e.g. resolution of 3.3 sigma up to 5 GeV from NIMA paper)

Why do the scrtucture combine in the way they do with the coefficents of cos phi terms and epsilons? Should I include the LT' sin phi term? What about the beam pols from andrey kim thesis?

Where to find clas12 data:
Run Group A - HallBWiki (jlab.org)
https://clasweb.jlab.org/wiki/index.php/Run_Group_A#tab=Trains

dpwg meeting page: https://clas12-docdb.jlab.org/cgi-bin/DocDB/private/DisplayMeeting?conferenceid=9



To Do for Thesis Talk:
Make plots of binning over kinematics eh not sure necessary if feels like now\\
Change refrences to include author names - eh probably not\\

 DVMP is sensitive to chiral odd GPDs, distinguishing it from DVCS as a GDP probe because why? Because something involving photon helicity and pion helicity, I forget exactly though
 
 - IF TIME: On Exclusivity cuts, show better and lable in terms of before and after cuts\\
- IF TIME: Possibly include radiative corrections\\
- IF TIME: 
- If TIME: Should include geometrical definition of phi - e.g. ppxppidiv something\\

What is Handbag approach / mechanism? When is it applicable?\\
t' stands for.. t - $t_0$ where $t_0 = \frac{-4m^2\xi^2}{1-\xi^2}$


In addition to collinear momentum distribution of partons inside the
nucleon, GPDs also encode the distribution of partons in the plane transverse to
the nucleons momentum in the infinite momentum frame [58]. Moreover, their
relation to energy-momentum tensor (EMT) form factors allow us to access the
EMT densities, the distribution of energy, angular momentum, pressure, and shear
forces inside the nucleon [15].

    Only valence quarks contribute electroproduction
of uncharged pions.

  The bracket $\langle \tilde{F} \rangle$ is the convolution of a GPD and an appropriate subprocess amplitude:
    $\langle \tilde{F} \rangle = \Sigma_{\lamda} \int_{-1}^{1} d\bar{x}H_{0\lambda,0\lambda}\left( \bar{x}, \xi, Q^2, t=0  \right)\tilde{F}\left( \bar{x}, \xi, Q^2, t  \right)\
    $ 
    
    Where $\lambda$ is the unobserved helicites of the partons participaing in the subprocess
    
 and epsilon is... 
    
    
    and t' stands for.. t - $t_0$ where $t_0 = \frac{-4m^2\xi^2}{1-\xi^2}$
    
    
    Where the skewness parameter is $\xi = \frac{x_B}{2-x_B}$ or whatever
       
    And k is the phase space factor given as 
     \scalebox{0.7}{%
     $           k = 16\pi \left( W^2 -m^2)\sqrt{\Lambda(W^2,-Q^2,m^2)} \right)$ 
    }
    
        Where $\Lambda(W^2,-Q^2,m^2)$ is the Källén function and $\mu_{pi}$ is the reduced pion mass given as 
       \scalebox{0.7}{%
      $         \mu_{\pi^0} = \frac{m^2_{\pi^0}}{m_u+m_d}$ 
    }
    $m_u$ and $m_d$ are respective masses of up and down quarks.
    
     Where $\Gamma (Q^2, x_B, E)$ is the virtual photon flux, is 
     \scalebox{0.7}{%
        $         \Gamma (Q^2, x_B, E) = \frac{\alpha}{8\pi} \frac{Q^2}{m^2_pE^2}\frac{1-x_B}{x_B^3}\frac{1}{1-\epsilon}$ 
    }

\definecolor{mygreen}{RGB}{14, 176, 9}
\definecolor{myyellow}{RGB}{204, 204, 10}
\definecolor{mypink}{RGB}{255, 51, 255}
\definecolor{mypurp}{RGB}{153, 21, 255}


\definecolor{lightred}{RGB}{255, 132, 145}
\definecolor{darkred}{RGB}{225, 0, 0}
\definecolor{lightorange}{RGB}{255, 200, 84}
\definecolor{darkorange}{RGB}{225, 150, 0}

\definecolor{lightgreen}{RGB}{85, 255, 91}
\definecolor{darkgreen}{RGB}{0, 170, 6}
\definecolor{lightblue}{RGB}{88, 200, 255}
\definecolor{darkblue}{RGB}{0, 81, 203}

% Chiral Even GPDs
\newcommand{\GPDH}{\textcolor{lightred}{${H}$}}
\newcommand{\GPDHEQ}{\textcolor{lightred}{{H}}}

\newcommand{\GPDE}{\textcolor{lightgreen}{${E}$}}
\newcommand{\GPDEEQ}{\textcolor{lightgreen}{{E}}}

\newcommand{\GPDHtilde}{\textcolor{lightorange}{$\tilde{H}$}}
\newcommand{\GPDHtildeEQ}{\textcolor{lightorange}{\tilde{H}}}

\newcommand{\GPDEtilde}{\textcolor{lightblue}{$\tilde{E}$}}
\newcommand{\GPDEtildeEQ}{\textcolor{lightblue}{\tilde{E}}}



%Chiral Odd GPDs

\newcommand{\GPDHT}{\textcolor{darkred}{$H_T$}}
\newcommand{\GPDHTEQ}{\textcolor{darkred}{H_T}}

\newcommand{\GPDET}{\textcolor{darkgreen}{$E_T$}}
\newcommand{\GPDETEQ}{\textcolor{darkgreen}{E_T}}

\newcommand{\GPDHTtilde}{\textcolor{darkorange}{$\tilde{H}_T$}}
\newcommand{\GPDHTtildeEQ}{\textcolor{darkorange}{\tilde{H}_T}}


\newcommand{\GPDETtilde}{\textcolor{darkblue}{$\tilde{E}_T$}}
\newcommand{\GPDETtildeEQ}{\textcolor{darkblue}{\tilde{E}_T}}


\newcommand{\GPDETbar}{\textcolor{mypurp}{$\bar{E}_T$}}
\newcommand{\GPDETbarEQ}{\textcolor{mypurp}{\bar{E}_T}}


There are 8 GPDs, 4 correspond to helicity conserving (chiral even) processes and 4 correspond to are helicity flipping (chiral odd) processes: \GPDH,  \GPDE,  \GPDHtilde,  and \GPDEtilde for chiral even, and \GPDHT,  \GPDET,  \GPDHTtilde, and \GPDETtilde \\
\GPDETbar = 2*\GPDHTtilde+\GPDET is commonly used

    \begin{tabular}{@{} *{4}{c} @{}}
    \headercell{Nucleon \\ Polarization} & \multicolumn{3}{c@{}}{Quark Polarization}\\
    \cmidrule(l){2-4}
    & U &  L & T    \\ 
    \midrule
      U  & \GPDH &                                   &  \GPDETbar \\
      L  &                    &  \GPDHtilde &                                   \\
      T  & \GPDE &                                   &  \GPDHT,\GPDHTtilde \\
    \end{tabular}

    %\vspace{-0.5cm}
    The cross section for DV$\pi^0$P has been theoretically linked to GPDs, which describe the 3D structure of the nucleon.


     \scalebox{0.7}{%
    $
         \frac{d^4\sigma_{\gamma^*p \rightarrow p'\pi^0}}{dQ^2dx_Bdtd\phi_{\pi}} =
         \Gamma (Q^2, x_B, E)
         \frac{1}{2\pi}
         \left\{ \left( \frac{d\sigma_T}{dt}+\epsilon\frac{d\sigma_L}{dt} \right)+ \epsilon cos(2\phi) \frac{d\sigma_{TT}}{dt} + \sqrt{2\epsilon(1+\epsilon)} cos(\phi) \frac{d\sigma_{LT}}{dt} \right\}
    $
    }
    
    Why do the scrtucture combine in the way they do with the coefficents of cos phi terms and epsilons?
    
    And the structure functions can be written as:

    \scalebox{0.7}{%   
    $      \frac{d\sigma_{L}}{dt} = 
    \frac{4\pi\alpha}{kQ^2}\left\{ \left( 1 - \xi^2 \right) 
    \lvert \langle \GPDHtildeEQ \rangle \rvert ^2 
    -2\xi^2 \Re \left[  \langle \GPDHtildeEQ \rangle ^* \langle \GPDEtildeEQ \rangle    \right] - \frac{t'}{4m^2}\xi^2
    \lvert \langle \GPDEtildeEQ \rangle \rvert ^2  \right\}$
    }\\
    
    \scalebox{0.7}{%   
    $      \frac{d\sigma_{T}}{dt} = 
    \frac{2\pi\alpha \mu_{\pi}^2}{kQ^4}
    \left\{ \left( 1 - \xi^2 \right) 
    \lvert \langle \GPDHTEQ \rangle \rvert ^2
    - \frac{t'}{8m^2}
    \lvert \langle \GPDETbarEQ \rangle \rvert ^2  \right\}$
    }\\
    
    \scalebox{0.7}{%   
    $      \frac{d\sigma_{LT}}{dt} = 
    \frac{4\pi\alpha \mu_{\pi}}{\sqrt{2}kQ^3}
    \xi\sqrt{1-\xi^2}
    \frac{\sqrt{-t'}}{2m}
    \Re \left\{ 
     \langle \GPDHTEQ \rangle ^*
    \langle \GPDEtildeEQ \rangle   
    \right\}$
    }\\
    
    \scalebox{0.7}{%   
    $      \frac{d\sigma_{TT}}{dt} = 
    \frac{4\pi\alpha \mu_{\pi}^2}{kQ^4}
    \frac{-t'}{16m^2}
    \langle \GPDETbarEQ \rangle^2   
    $
    }\\
    
    Only valence quarks contribute electroproduction
of uncharged pions.

    and epsilon is... 
    
    
    and t' stands for.. t - $t_0$ where $t_0 = \frac{-4m^2\xi^2}{1-\xi^2}$
    
    
    Where the skewness parameter is $\xi = \frac{x_B}{2-x_B}$ or whatever
    
    The bracket $\langle \tilde{F} \rangle$ is the convolution of a GPD and an appropriate subprocess amplitude:
    $\langle \tilde{F} \rangle = \Sigma_{\lamda} \int_{-1}^{1} d\bar{x}H_{0\lambda,0\lambda}\left( \bar{x}, \xi, Q^2, t=0  \right)\tilde{F}\left( \bar{x}, \xi, Q^2, t  \right)\
    $ 
    
    Where $\lambda$ is the unobserved helicites of the partons participaing in the subprocess
    
    
    
    Where $\Gamma (Q^2, x_B, E)$ is the virtual photon flux, is 
     \scalebox{0.7}{%
        $         \Gamma (Q^2, x_B, E) = \frac{\alpha}{8\pi} \frac{Q^2}{m^2_pE^2}\frac{1-x_B}{x_B^3}\frac{1}{1-\epsilon}$ 
    }
    
    And k is the phase space factor given as 
     \scalebox{0.7}{%
     $           k = 16\pi \left( W^2 -m^2)\sqrt{\Lambda(W^2,-Q^2,m^2)} \right)$ 
    }
    
        Where $\Lambda(W^2,-Q^2,m^2)$ is the Källén function and $\mu_{pi}$ is the reduced pion mass given as 
       \scalebox{0.7}{%
      $         \mu_{\pi^0} = \frac{m^2_{\pi^0}}{m_u+m_d}$ 
    }
    $m_u$ and $m_d$ are respective masses of up and down quarks.

    \fi