This work analyzes data obtained in the Fall 2018 run under the officially approved JLab project E12-06-108, which had a stated goal of measuring the electroproduction \xsecs for pseudoscalar mesons ($\pi^0$ and $\eta$) \parencite{Avakian2011HardCLAS12}. The experiment's design luminosity was 10$^{35}$ cm$^{-2}$ s$^{-1}$ corresponding to 75 nA on the LH2 target, but ran at a slightly adjusted beam current of 50 nA. Data was taken with the torodial magnetic field at both possible polarities - ``outbending'' or ``+1'' corresponding to bending the scattered electron away from the beamline, and ``inbending'' or ``-1'' corresponding to bending it towards the beamline. 

Data was taken at a beam energy of 10.6 GeV with a beam current of approximately 50 nA and a beam polarization of 86.9\%. The data taking runs exceeded the original design DAQ requirements of a 10kHz event rate with 100 MB/s data rate to 12 kHz event rate with 550 MB/s data rate. The integrated luminosity was 5.5\sci{40} inverse femtobarns for the inbending dataset and 4.7\sci{40} inverse femtobarns for the outbending configuration, yielding more than 1.2 PB of raw data combined. \todo{fact cite the raw data amount}


\iffalse
%EXTRA INFORMATION

The 499 MHz beam structure equates to a signal with a period of 2.004 ns. In practice, the beam was delivered in every other RF bucket, and so bunched at a period of 4.008 ns.

       
            CLAS12 runs with "open trigger", which means different sub-experiments can define their own triggering logic. There is a standard electron trigger, based off of hits in HTCC, ECal, and FTOF. 
Only about 50\% of the electron triggers recorded with an inbending torus polarity are actually electrons. For outbending torus polarity, hte electron trigger purity is as high at 70\%. 

%Mention configurations and combinatorics


1 of 13 proposed experiments in the group named "RGA"
• Different experiments have different patterns in CLAS12
Trigger decision based on PMT detectors and DC, configured for 3 sets of experiments: “electrons”, “MesonEx”, “muons”
• Original DAQ requirements: 10kHz event rate, 100MB/sec data rate, LT= 0.9
• Production rates at 50nA beam, FT=ON: 12kHz event rate, 550MB/sec data rate, LT=0.94%
Electron trigger 5kHz (40%), muon trigger 2.7kHz (20%), MesoEx trigger 4.5kHz (40%)
• Data Reconstruction: 342 of the accumulated statistics ready for physics analyses (cooked data)
60G triggers, 1.2 PB raw data → 160TB DST → 100TB DST (16M core/hours processing time, 62.5M jobs processed)
• Preparing a better version of data cooking (Pass2) to recook the entire data set
Pass1 data suffer from CD mis-alignment and other issues solved in the new cooking


   % \href{https://www.jlab.org/Hall-B/clas12-web/}{Detector Specs}
    
   % 20 kHz Level 1 trigger rate, 1 GB/s.


        
\fi




