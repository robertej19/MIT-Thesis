
The 499 MHz beam structure equates to a signal with a period of 2.004 ns. In practice, the beam was delivered in every other RF bucket, and so bunched at a period of 4.008 ns.

       
            CLAS12 runs with "open trigger", which means different sub-experiments can define their own triggering logic. There is a standard electron trigger, based off of hits in HTCC, ECal, and FTOF. 

        Only about 50\% of the electron triggers recorded with an inbending torus polarity are actually electrons. For outbending torus polarity, hte electron trigger purity is as high at 70\%. 
    
    \href{https://www.jlab.org/Hall-B/clas12-web/}{Detector Specs}
    
    20 kHz Level 1 trigger rate, 1 GB/s.

   The CLAS12 detector is a large angle spectrometer that generally covers angles from 5 to 130 degrees, spanned by two main detector subsystems - the Forward Detector and the Central Detector.

    
   CLAS12 at the design luminosity of 1035 cm−2 s−1

        Data taken is RGA taken in Fall 2018

        Mention configurations and combinatorics


%from sangbaek
In fall 2018, two sets of experiments have been performed with opposite toroidal magnetic field directions
keeping the other detector settings the same. The toroidal magnet bends the scattered
trigger electron inward or outward along the beam direction. In convention, the torus
polarity associated with the inwardly bending electrons is called the negative, -1,
-100\%, or inbending polarity. The opposite is called the positive, +1, +100\%, or
outbending polarity. Both experiments took data with the beam energy of 10.6 GeV,
and the beam current of about 50 nA. The effects due to the variation in the beam
current during the run periods will be taken into account at the end of the analysis.
The CLAS collaboration performed other CLAS12 experiments with different targets
like liquid deuterium, and various beam energies. The description in this thesis
will be focused on the RG-A fall 2018.

The measured
electron beam polarization was 86.9\% during the RG-A data taking in fall 2018.
