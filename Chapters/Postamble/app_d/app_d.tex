\chapter{Notes on Kinematic Relationships}


Derivation of infinite momentum frame Bjorken x. Take quark to have momentum fraction $\xi$ of proton's total momentum,i.e. $p_q = \xi p_2$:\\
            \indent Inf. Mom. frame - neglect proton mass so $p_2$ = $E_2$, neglect all transverse momenta:\\
            \indent Struck quark 4-momenta: $p_q = \xi p_2 = (\xi E_2, \xi E_2, 0, 0)$\\
            \indent 4-momenta of quark after interaction: $(p_q + q) = (\xi p_2 +q)$\\
            \indent Square the 4-momenta $(\xi p_2 +q)^2 = \xi^2 p_2^2 +q^2 + 2\xi p_2 \cdot q  = m_q^2$\\
            \indent Continue, noting $p_q = \xi p_2$ : $m_q^2 = p_q^2 - Q^2 + 2 \xi p_2 \cdot q$\\
            \indent Since $p_q^2$ = $m_q^2$, we have: $m_q^2 = m_q^2 -Q^2 + 2 \xi p_2 \cdot q$\\
            \indent So $0 = -Q^2 + 2\xi p_2 \cdot q \longrightarrow \xi = \frac{Q^2}{2 p_2 \cdot q} = x_B$\\


\begin{figure}
    \centering
    \rotatebox{270}{\includegraphics[width=0.99\textwidth]{Chapters/Postamble/app_d/pics/analysis-notes.jpg}}
    \caption{Caption}
    \label{fig:enter-label}
\end{figure}